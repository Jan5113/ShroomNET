\section*{Abstrakt}
Diese Maturarbeit beschäftigt sich mit der Entwicklung und Optimierung eines Neuronalen Netzes für die Bestimmung von Pilzarten. Ziel der Arbeit ist es, die Umsetzbarkeit eines solchen Neuronalen Netzes zu prüfen. Zur Vereinfachung beschränkt sich die Erkennung vorerst auf die 20 häufigsten Pilzarten der Nordwestschweiz.

Im Theorieteil der Dokumentation soll ein grundlegendes Verständnis für \textit{Neuronale Netze} und \textit{Deep Learning} vermittelt werden, um die darauf folgende Dokumentation der Umsetzung nachvollziehbar gestalten zu können.

Bei der Umsetzung des Algorithmus werden verschiedene Techniken und Vorgehensweisen für den Aufbau und das Training des Neuronalen Netzes in Betrachtung gezogen, um eine möglichst akkurate Bestimmung der Pilzart ermöglichen zu können. Dabei werden verschiedene Netzwerkarchitekturen wie auch Datenaufbereitungsmethoden gegeneinander abgewägt und auf deren spezifische Vor- und Nachteile untersucht.

In einer Zusammenfassung soll schliesslich über die Anwendbarkeit, Zuverlässigkeit und Erweiterbarkeit des Algorithmus diskutiert werden. 
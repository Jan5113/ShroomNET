\section{Einführung}

Wer behaupten will, er habe noch nie von \textit{künstliche neuronalen Netzen} gehört zu haben, muss gewissermassen unter einem Stein leben. \textit{Künstliche neuronale Netze} (kurz \textit{KNN}) haben in den letzten Jahren ein Comeback erlebt und haben es auch schon einige Male in die Schlagzeilen geschafft:
\vspace{0.5cm}

\begin{center}
	{\LARGE \textit{''Googles AlphaGo KI \footnote{Künstliche Intelligenz: Modellierung/Nachahmung von ''intelligentem'' Verhalten durch Computer, z.B. Schachcomputer, Übersetzungs-Tools oder Text-/Spracherkennung} besiegt}\\}
	\vspace{0.5cm}
	{\LARGE \textit{Go-Weltmeister Ke Jie''}}\cite{theverge}
\end{center}

\vspace{0.5cm}


\begin{minipage}{13.5cm}
Das chinesische Brettspiel \textit{Go} basiert auf ganz simplen Regeln: Schwarze bzw. weisse Spielsteine werden abwechslungsweise auf ein 19x19 Felder grosses Spielfeld gesetzt. Das Ziel des Spiels ist es, den Gegner einzukreisen und zu erobern \cite{howtogo}. Einfache Regeln, jedoch fast unzählige Wege: Insgesamt gibt es rund $10^{170}$ verschiedene Möglichkeiten, die Steine anzuordnen. Ein kurzer Magnitudenvergleich: Schach hat ''nur'' etwa $10^{47}$ verschiedene Anordnungsmöglichkeiten \cite{shannon}, das beobachtbare Universum hat ''nur'' $10^{80}$ Atome \cite{atoms}.

Diese schiere Anzahl an Möglichkeiten machten \textit{Go} bei KI-Programmierern sehr hoch angesehen, denn im Gegensatz zu Schach  können nicht alle relevanten Züge in absehbarer Zeit vorausberechnet werden. Trainierte Go-Spieler können das auch nicht, dafür verlassen sie sich auf ihre Erfahrung und auf ihr Intuition, etwas, was Computern nur schwierig beizubringen ist. Und doch passierte es im Mai 2017: Der Go-Weltmeister Ke Jie wurde geschlagen --- von der \textit{KI AlphaGo} \cite{alphago}.
\end{minipage}

\vspace{0.5cm}

\textit{AlphaGo} unterscheidet sich grundlegend von konventionellen Spiele-KIs, weswegen es auch in \textit{Go} gegen Ke Jie antreten konnte. Der Algorithmus hinter \textit{AlphaGo} basiert auf \textit{künstlichen Neuronalen Netzen} und \textit{Deep Learning}. Diese Algorithmen sind in der Lage, aus Beispielen Muster zu erkennen, zu lernen und sich selber dadurch zu verbessern. Durch die Analyse von zahlreichen von Menschen gespielten Partien lernte \textit{AlphaGo} die Grundlagen und Grundstrategien des Spiels. Daraufhin liess man den Algorithmus gegen sich selber spielen, wodurch \textit{AlphaGo} sich über das Niveau des menschlichen Weltmeisters hinaus begab. 


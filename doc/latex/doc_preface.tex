\section*{Vorwort}
 Für \textit{künstliche neuronale Netze} hatte ich schon immer eine gewisse Faszination. Ein im Grunde genommen statischer Code modelliert organisches Verhalten: Er kann lernen und sich dadurch verbessern. Und wie wir auch nicht genau wissen, wie unser Gehirn funktioniert, so wissen wir es auch nicht von den \textit{neuronalen Netzen} --- sie \textit{sind} gewissermassen ein ''elektronisches Hirn''. Weiss man aber dieses ''elektronische Hirn'' zu belehren und einzusetzen, so eröffnen sich grenzenlose Möglichkeiten.\\
 
 Codezeilen haben mich schon lange begleitet; die ersten Schritte machte ich mit meinem Vater: Variablen definieren, Werte daraus berechnen, erste Schleifen. Mit der Mittelschule und dem \textit{infcom}-Kurs kamen Objekte, die Steuerung von grafischen Elementen und Game-Loops hinzu. Nebenbei hielt ich mich mit der aktuellen Technik auf dem Laufenden, welche aber zu viel mehr imstande war: Bilder und Sprache erkennen, später lernten sie auch, den Menschen in den komplexesten Brettspielen\cite{alphago} und Computerspielen\cite{openai} zu schlagen. Weiter erschienen absurde Reden von Staatsoberhäuptern, bei denen das Auge nicht mehr von echt oder gefälscht unterscheiden konnte \cite{deepfake}. Diese und weitere Möglichkeiten, aber auch die Genialität dieser Algorithmen haben mein Interesse an den \textit{künstlichen neuronalen Netzen} geweckt, jedoch fehlten mir damals die Programmiergrundlagen, um ein solches Projekt umzusetzen. Mit der ersten grösseren Arbeit ''EvoSim''\cite{evosim} habe ich meine bisherigen Programmierkenntnisse festigen und erweitern können. Zudem habe ich mich in das Themengebiet der sogenannten \textit{genetischen Algorithmen} begeben; eine rein durch Zufall und Selektion vorangetriebene Methode des \textit{maschinellen Lernens}. Mit der Maturarbeit will ich somit einen Schritt weiter gehen und den Einstieg in die komplexere Welt der \textit{künstlichen neuronalen Netze}, den ''elektronischen Hirnen'', machen.\\
 
 Die Anwendung von \textit{künstlichen neuronalen Netzen} kann sehr vielfältig sein; enorm stark vertreten sind sie im Bereich der Bilderkennung. Da viele bekannte Problemstellungen wie Gesichts- und Handschrifterkennung schon zu genüge behandelt worden sind, habe ich mich auf den Impuls von Dr. Nicolas Ruh auf ein für mich eher exotischeres Anwendungsgebiet eingelassen: die Erkennung von Pilzarten. Es ist bekannt, dass die zuverlässige Bestimmung von Pilzen langjährige Expertise voraussetzt, da sich Arten zum Teil nur anhand von wenigen Details auseinanderhalten lassen. Liesse sich dieser Prozess zu einem gewissen Teil von \textit{neuronalen Netzen} übernehmen, so könnte man diese ''Expertise'' für jedermann mit einem Mobiltelefon in der Tasche zugänglich und nutzbar machen.\\ 
 
 Daraus entstand der konkrete Plan, einen auf \textit{künstlichen neuronalen Netzen} basierenden Algorithmus zu entwickeln, welcher für die Pilzartenbestimmung ausgelegt ist --- daher auch der Name \textit{ShroomNET}. Der Erkennungsalgorithmus soll der zentrale Punkt dieser Arbeit sein, wobei der Fokus auf die Entwicklung und Optimierung gelegt wird.\\
 
 Während den Recherchen für diese Arbeit fanden wir ein Team von Studenten der Universität Helsinki, welches für ihr Projekt namens ''Deep Shrooms''\cite{deepshroom} ein ähnliches Konzept hatten. Jedoch war das Projekt leider erfolglos, weswegen ich in dieser Arbeit die selbe Hürde umso mehr zu nehmen versuchen will.
 
 
 